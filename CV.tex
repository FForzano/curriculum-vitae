\documentclass[
    italian,
    a4paper,
    nologo,
    notitle
]{europasscv}

\usepackage{helvet}
\usepackage{xcolor}
\usepackage{enumitem}

\usepackage[main=italian, english]{babel}

\newcommand{\englishText}[1]{\foreignlanguage{english}{#1}}

\newlist{freeitemize}{itemize}{1}
\setlist[freeitemize,1]{ label={}, left=0pt, itemsep=0.2cm, partopsep=0pt,
    parsep=0pt, topsep=0pt, listparindent=0pt, labelwidth=0pt, labelsep=0pt, before=\vspace{-0.33cm}}

% \definecolor{ecvrulecolor}{}{}
% \definecolor{ecvsectioncolor}{}{}
% \definecolor{ecvhighlightcolor}{}{}
% \definecolor{ecvhighlightcellcolor}{}{}
% \definecolor{ecvtablebordercolor}{}{}
% \definecolor{ecvlanglinkcolor}{}{}
% \definecolor{ecvtextcolor}{}{}

\ecvname{Federico Forzano}
\ecvaddress{Via Benvenuto Tisi da Garofalo, 15, 44121 Ferrara (FE), Italia}
\ecvmobile{(+39) 3281772074}
\ecvemail{f.forzano99@gmail.com f.forzano@pec.it}
\ecvgithubpage{https://github.com/FForzano}
\ecvdateofbirth{15-12-1999}
\ecvnationality{Italiana}
\ecvgender{Uomo}

\begin{document}
\begin{center}
    \fontfamily{phv}\selectfont
    \textcolor{ecvtextcolor}{{\huge Curriculum Vitae}}
    \vspace{0.7cm}
\end{center}
\begin{europasscv}
    \ecvpersonalinfo

    % Section: Formazione
    \ecvsection{Istruzione e formazione}
    % Ph.D. Student
    \ecvtitle{11/2023 - oggi}{Dottorato di Ricerca}
    \ecvitem{Istituto o Università}{Università degli Studi di Ferrara}
    \ecvitem{Descrizione}{Sono attualmente un dottorando presso il dipartimento
        di ingegneria dell'università di Ferrara, come parte del
        \textbf{\englishText{Wireless Communication and Localization Networks
                (WCLN) Laboratory}}. I miei interessi di ricerca riguardano le \englishText{quantum information science} e, in
        particolare, il \englishText{quantum sensing}. }

    % Università
    \ecvtitle{09/2021 - 10/2023}{Laurea Magistrale}
    \ecvitem{Qualifica conseguita}{Laurea Magistrale in \emph{Ingegneria
            Elettronica per l'ICT}}
    \ecvitem{Istituto o Università}{Università degli Studi di Ferrara}
    \ecvitem{Titolo della tesi}{\emph{\englishText{Analysis of quantum
                illumination systems}}}
    \ecvitem{Votazione}{110/110 e lode}
    \ecvitem{Esami significativi} {\textbf{Tecniche di decisione, stima e
            sensing distribuito} (statistica inferenziale, teorie della
        decisione e della stima Bayesiane e non); \textbf{Ecosistemi
            wireless} (modulazioni con memoria CPM, ricevitori a MLSE, canale
        wireless, segnali multiportante, sistemi MIMO e tecniche di
        diversità); \textbf{Informazione e codici} (teoria dell'informazione
        di Shannon, fondamenti di teoria dei codici con focus sui codici a
        blocco lineari); \textbf{Elettronica dei sistemi wireless}
        (architetture dei \englishText{tranceiver}, dispositivi a
        semiconduttore per l'alta frequenza, amplificatori lineari ad alta
        frequenza e LNA, mixer, VCO, PLL e amplificatori di trasmissione);
        \textbf{Propagazione guidata} (propagazione elettromagnetica in
        guide metalliche, risuonatori, guide slab e fibre ottiche)}.

    \ecvtitle{09/2018 - 10/2021}{Laurea Triennale}
    \ecvitem{Qualifica conseguita}{Laurea Triennale in \emph{Ingegneria
            Elettronica e Informatica}}
    \ecvitem{Istituto o Università}{Università degli Studi di Ferrara}
    \ecvitem{Titolo della tesi}{\emph{\englishText{On the Design of Quantum Communication
                Systems with non-Gaussian States}}}
    \ecvitem{Votazione}{110/110 e lode}
    \ecvitem{Esami significativi} {\textbf{Segnali e comunicazioni} (analisi di
        Fourier per segnali tempo invarianti, sistemi LTI, modulazioni senza
        memoria, analisi di segnali aleatori); \textbf{Metodi statistici per
            l'ingegneria} (calcolo combinatorio e teoria della
        probabilità); \textbf{Sistemi wireless} (sistemi di comunicazione
        passa-basso e passa-banda, modulazione e demodulazione di segnali
        digitali, non idealità del canale wireless); \textbf{Reti di
            telecomunicazione e internet} (sistemi a coda, algoritmi di accesso
        multiplo, algoritmi di routing, fondamenti di internetworking);
        \textbf{Sistemi elettronici analogici} (amplificatori per piccoli
        segnali, amplificatori per grandi segnali, convertitori AC/DC e
        raddrizzatori DC/DC, OP-AMP, oscillatori e multivibratori)}.
    \newpage
    % Scuole superiori
    \ecvtitle{2013 - 2018}{Scuole Superiori}
    \ecvitem{Qualifica conseguita}{Diploma di Maturità Scientifica - Scienze Applicate}
    \ecvitem{Istituto o Università}{Liceo Scientifico Statale "A. Roiti", Ferrara}
    % \ecvitem{Votazione}{75/100}    

    % Section: Esperienze lavorative
    \ecvsection{Esperienze lavorative}
    % FPC DIDATTICA 4.0 S.R.L.
    \ecvtitle{2023 - oggi}{Socio fondatore}
    \ecvitem{Azienda o Ente}{FPC DIDATTICA 4.0 S.R.L.}
    \ecvitem{Descrizione}{Dall'esperienza come insegnante privato svolta nella città di Ferrara, è nata l'idea di avviare una
        società con l'obiettivo di creare un ponte tra insegnanti qualificati e
        famiglie di tutta Italia. \textbf{FPC DIDATTICA 4.0 S.R.L.}
        si pone come garante di qualità e di lezioni "in regola" tra le parti e fornisce tutti gli strumenti necessari per agevolare la
        didattica.
        In tale contesto prendo parte alla gestione dell'azienda come socio e
        membro del CdA e mi occupo di coordinare gli aspetti tecnici e di
        sviluppo.}
    % Formando PerCorsi
    \ecvtitle{03/2019 - 11/2023}{Insegnante privato}
    \ecvitem{Azienda o Ente}{Formando PerCorsi di Giovanni Govoni}
    \ecvitem{Descrizione}{Durante l'intera durata dei miei studi universitari ho
        lavorato come insegnante privato per studenti delle scuole superiori e
        università. L'intera attività è stata svolta in veste di lavoratore
        autonomo in collaborazione con \textbf{Formando PerCorsi di Giovanni Govoni}.
        Negli corso di questi anni ho avuto la possibilità di affiancare più di 50
        studenti impegnando un monte orario compreso fra le 10 e le 15 ore
        settimanali.}

    % Section: Partecipazioni a conferenze
    \ecvsection{Partecipazioni a conferenze}
    % IEEE InfoCom 2025
    \ecvtitle{19/05/2025}{\englishText{IEEE International Conference on Computer Communications
        (InfoCom) 2025}}
        \ecvitem{Luogo}{London, Regno Unito}
    \ecvitem{Titolo}{\englishText{Quadrature Measurement Characterization for
        Single-Mode Photon-Varied Gaussian States}}
    \ecvitem{Descrizione}{Ho partecipato come autore al workshop \textbf{Quantum Networked Applications and Protocols (QuNAP)} 
        della conferenza \textbf{IEEE
            InfoCom 2025} con il contributo dal titolo
        \emph{\englishText{Quadrature Measurement Characterization for
        Single-Mode Photon-Varied Gaussian States}}.}

    % Section: Altre esperienze
    \ecvsection{Altre esperienze}
    % Tutorato didattico di segnali
    \ecvtitle{2025}{Tutorato didattico}
    \ecvitem{Descrizione}{Preparazione e svolgimento di attività di tutorato
    didattico di supporto alla didattica per l'insegnamento di
    \textbf{Probabilità, Statistica e Segnali} del corso di laurea in Ingegneria
    Elettronica e Informatica dell'Università degli Studi di Ferrara.}
    % Laboratori didattici
    \ecvtitle{2024 - oggi}{Laboratori didattici}
    \ecvitem{Descrizione}{Preparazione e svolgimento di laboratori didattici per
        l'insegnamento di \textbf{Reti di telecomunicazione e internet} dei corsi di
        laurea in Ingegneria Elettronica e Informatica e in Informatica,
        dell'Università degli Studi di Ferrara.}
    % Tirocinio
    \ecvtitle{2023 - oggi}{Correlatore di tesi}
    \ecvitem{Descrizione}{ Durante il mio dottorato di ricerca, ho avuto la
        possibilità di affiancare alcuni studenti nei loro tirocini interni
        all'università e di essere correlatore delle loro tesi.}
    \ecvitem{Tesi}{\begin{freeitemize}%
            \item A. Balotta, ``Sviluppo di esperienze didattiche per reti di
            comunicazione,'' B.S. Thesis, Dept. Eng., Univ. Ferrara, Ferrara,
            Italy, 2024. Supervisor: Prof.~A.\,Conti; Co-supervisor: F.\,Forzano.
            \item A. Calò, ``Denoising di segnali EEG per interfacce
            cervello-computer,'' B.S. Thesis, Dept. Eng., Univ. Ferrara,
            Ferrara, Italy, 2024. Supervisor: Prof.~A.\,Conti; Co-supervisor: F.\,
            Forzano.
        \end{freeitemize}}

    % Section: Competenze personali
    \ecvsection{Competenze tecniche\\e professionali}
    \ecvitem{Sistemi operativi}{
        \begin{freeitemize}[itemsep=0pt]
            \item \textbf{Linux:} Conoscenza avanzata sia in ambito desktop che server.
            \item \textbf{Windows:} Buona conoscenza in ambito desktop.
            \item \textbf{macOS:} Buona conoscenza in ambito desktop.
        \end{freeitemize}
    }
    \ecvitem[0.3cm]{Linguaggi di programmazione}{
        \begin{freeitemize}[itemsep=0pt]
            \item \textbf{Python:} Conoscenza avanzata per applicazioni di
            calcolo numerico (svolti progetti su varie tematiche, tra cui
            principal component analysis (PCM), continuous phase modulations
            (PCM) e autoencoders convoluzionali su segnali tempo-varianti),
            intermedia per sviluppo di applicazioni web.
            \item \textbf{C:} Buona conoscenza e padronanza del linguaggio.
            \item \textbf{Java:} Conoscenza intermedia.
            \item \textbf{Matlab:} Conoscenza avanzata.
            \item \textbf{PHP:} Conoscenza avanzata del linguaggio e del
            \englishText{framework} Yii2.
            \item \textbf{JavaScript/TypeScript:} Conoscenza intermedia dei
            linguaggi e della libreria React.
            \item \textbf{HTML/CSS:} Conoscenza intermedia.
            \item \textbf{SQL:} Conoscenza intermedia.
            \item \textbf{VHDL:} Conoscenza base.
            \item \textbf{LaTeX:} Conoscenza avanzata per la stesura di
            documenti e padronanza del pacchetto TikZ per la creazione di
            immagini e grafici.
            \item \textbf{Bash:} Conoscenza intermedia.
        \end{freeitemize}
    }
    \ecvitem[0.3cm]{Tecnologie e strumenti digitali}{
        \begin{freeitemize}[itemsep=0pt]
            \item \textbf{Git/Github/Jira:} Conoscenza avanzata per la gestione
            di progetti software e per il suo \englishText{deployment}.
            \item \textbf{KVM/QEMU:} Conoscenza avanzata.
            \item \textbf{Docker:} Conoscenza intermedia.
            \item \textbf{Wireshark:} Conoscenza intermedia.
            \item \textbf{Apache/Nginx:} Esperienza nella configurazione e
            gestione di server web e di \englishText{reverse proxy} con tali
            software.
            \item \textbf{IPTables:} Esperienza base nella configurazione di
            iptables per la creazione di firewalls e per la configurazione di
            NAT.
            \item \textbf{FRR:} Esperienza base nella configurazione di FRR per
            il routing.
        \end{freeitemize}
    }

    \ecvsection{Competenze linguistiche}
    \ecvmothertongue{Italiano}
    \ecvlanguageheader
    \ecvlanguage{Inglese}{B2}{B2}{B2}{B1}{B1}
    \ecvlanguagefooter
\end{europasscv}
\vspace{1cm}
\newpage
\textcolor{ecvtextcolor}{
    \begin{itemize}
        \fontfamily{phv}\selectfont
        \item DICHIARAZIONE SOSTITUTIVA DI CERTIFICAZIONE \textbf{(art. 46 e 47 D.P.R. 445/2000)}\\
              Il sottocritto \textbf{Federico Forzano}, ai sensi e per gli effetti degli \textbf{articoli 46} e \textbf{47} e
              consapevole delle sanzioni penali previste dall'\textbf{articolo 76 del D.P.R. 28 dicembre 2000, n. 445} nelle ipotesi di
              falsità in atti e dichiarazione mendace, dichiara che le informazioni riportate nel presente curriculum vitae
              corrispondono a verità.
              \begin{flushright}
                  \textbf{Data e firma}\vspace{0.5cm}\\
                  \rule{0.2\textwidth}{0.4pt}, \rule{0.3\textwidth}{0.4pt}
              \end{flushright}
        \item Il sottoscritto dichiara di essere informato, ai sensi del \textbf{d.lgs. n.196/2003} e
              del \textbf{GDPR 679/16 - \emph{Regolamento europeo sulla protezione dei dati personali}} che i dati personali raccolti saranno trattati anche
              con strumenti informatici esclusivamente nell'ambito del procedimento per il
              quale la presente dichiarazione viene resa e per tutti gli adempimenti connessi.
              \begin{flushright}
                  \textbf{Data e firma}\vspace{0.5cm}\\
                  \rule{0.2\textwidth}{0.4pt}, \rule{0.3\textwidth}{0.4pt}
              \end{flushright}
    \end{itemize}
}
\end{document}